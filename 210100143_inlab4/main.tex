\documentclass{beamer}
\usepackage[utf8]{inputenc}
\usepackage{amsmath}
\usepackage{hyperref}
\hypersetup{
    colorlinks=true,
    linkcolor=blue,
    filecolor=magenta,      
    urlcolor=cyan,
    pdftitle={Slides},
    pdfpagemode=FullScreen,
    }

\usetheme{Madrid}
\usecolortheme{default}

\title[\LaTeX{} Basics \& Advanced] 
{\LaTeX{} Basics \& Advanced}


\author[Shubham Hazra] 
{Shubham Hazra}

\institute[IITB] 
{
  IIT Bombay
}

\date[2022]
{August 2022}

\logo{\includegraphics[height=1cm]{iitb.png}}

\begin{document}

\frame{\titlepage}
\begin{frame}
    \frametitle{Introduction of myself}
    I am Shubham Hazra, a second year undergraduate at the department of Computer Science
    at Indian Institute of Technology Bombay. I am required to make this beamer
    presentation and write my introduction on this slide as the part of the inlab 4 
    assignment of CS-251, Software Systems Lab course. Till now I am really enjoying
    this inlab as it is of practical use to me and is also relatively easier than the
    previous labs on sed and awk or git. I thought of using Lipsum text to fill this 
    section of slide but then just decided on writing a few lines by myself just so
    it looks a bit different from the template output pdf and what the other 
    people are writing. I am using TeXLive with VS Code for this instead of overleaf.
\end{frame}

\begin{frame}
    \frametitle{Table of Contents}
    \tableofcontents
    Note how the links here are redirecting to the corresponding page
\end{frame}

\section[Introduction]{Introduction}
\begin{frame}
    \frametitle{Introduction}
    We first see the power of frames in \textbf{\LaTeX}. We dont need to write each
    and every slide just for a new line.\pause We can just use beamer class with the
    feature of pauses.\pause However, \textbf{\LaTeX} has another ( rather the most important
    usage ), namely the use \alert{formatting text} in a more mathematical way.
\end{frame}

\section[Equations]{Equations}
\begin{frame}
    \frametitle{Equations}
    We can write many equations, can be labelled like the following
    \begin{equation}
        e^{i\alpha} = cos(\alpha) + i\,sin(\alpha)
    \end{equation}\pause
    or the unlabelled equations like the force between two charges given by
    \begin{equation*}
        F = \frac{1}{4\pi\epsilon_0}\frac{q_1q_2}{r^2}
    \end{equation*}
\end{frame}

\section[Itemize and Linking]{Itemize and Linking}
\begin{frame}
    \frametitle{Itemize and Linking}
    Also, \LaTeX can be used to present the items in a list format, for example,
    some common ways of sorting an array are:
    \begin{itemize}
        \item Bubble sort
        \item Insertion sort \pause, then there are the more rigorous algorithms like
        \item QuickSort
        \item Heap sort \pause, \textit{and then the best known algorithm}
        \item \alert{Monkey sort} (or) Bogo-sort.
    \end{itemize}
    Some pointers to the last algorithm can be found at \href{https://en.wikipedia.org/wiki/Bogosort}{here}
\end{frame}

\section[Matrices]{Matrices}
\begin{frame}
    \frametitle{Matrices}
    We can also write matrices in \LaTeX, for example the identity matrix of size
    (3x3) is\\
    \begin{center}
        \begin{math}
            I_3 = 
            \begin{bmatrix}
                1 & 0 & 0\\
                0 & 1 & 0\\
                0 & 0 & 1
            \end{bmatrix} 
        \end{math}
    \end{center}
    \pause
    \alert{Bonus: try to indent like the below equation}
    \begin{align*}
        \mathbf{(\textbf{a} \cdot \textbf{b})}^2 &= (\sum{a_ib_i})^2\\
        &\leq (\sum{a_i^2})(\sum{b_i^2})
    \end{align*}
\end{frame}

\end{document}